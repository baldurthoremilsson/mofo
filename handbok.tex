%%
%%  Beinagrind fyrir Morpho handbók í LaTeX.
%%  Til að keyra þetta gegnum LaTeX forritið
%%  má t.d. nota pdflatex í cygwin með eftir-
%%  farandi skipun í bash:
%%
%%       pdflatex handbok
%%   eða (virkar líka í cmd):
%%       bash -c 'pdflatex handbok'
%%
\documentclass[12pt,a4paper]{article}
\usepackage[utf8]{inputenc}
\usepackage[icelandic]{babel}
\usepackage[pdftex]{hyperref}
%\usepackage{makeidx,smplindx,fancyhdr,graphicx,times,multicol,comment}
\usepackage{times}
\usepackage[T1]{fontenc}
\usepackage[rounded]{syntax}
\newenvironment{repnull}[0]{%
	\begin{stack}
	\\
	\begin{rep}
}{%
	\end{rep}
	\end{stack}
}
\newenvironment{málrit}[1]{%
	\par\noindent\begin{minipage}{\linewidth}\vspace{0.5em}\begin{quote}\noindent%
	\hspace*{-2em}\synt{#1}:\hfill\par%
	\noindent%
	\begin{minipage}{\linewidth}\begin{syntdiag}%
}{%
	\end{syntdiag}\end{minipage}\end{quote}\end{minipage}%
}

\begin{document}
\sloppy
\title{Handbók fyrir {\bf{\tt mofo}}}
\author{Baldur Þór Emilsson}
\maketitle

\begin{abstract}
Útdráttur...
\end{abstract}

\tableofcontents

\section{Inngangur}
Mofo er forritunarmál sem þróað var í áfanganum Þýðendur í Háskóla Íslands vorið 2010.
Það byggir á málinu Morpho (http://morpho.cs.hi.is) eftir Snorra Agnarsson, en sækir einnig nokkuð
úr hugmyndafræði Python.
Mofo er mjög frumstætt og takmarkað þegar að útfærslumöguleikum kemur, en tenging þess í Morpho eykur notagildi þess til muna.
\section{Notkun og uppsetning}
Grunnkóða þýðandans má nálgast á http://www.baldur.biz/mofo/mofolang.zip og með honum fylgja Bash skriftur fyrir
þýðanda og keyrsluumhverfi. Í zip skránni er einnig að finna umhverfi fyrir Morpho, svo notandi þarf einungis Java
til að keyra Mofo forrit.

Mofo notar Flex og Bison í þýðandanum og er skrifað í C++. Til að þýða hann á Linux er hægt að nota {\tt make} skipunina.

Mofo kóði er geymdur í skrám sem enda á .mof. 
\section{Málfræði}
\subsection{Frumeiningar málsins}
\subsubsection{Athugasemdir}
\subsubsection{Lykilorð}
\subsection{Mállýsing}
\section{Merking málsins}
\subsection{Einingar og einingaraðgerðir}
\subsection{Gildi}
\subsection{Breytur}
\subsection{Merking segða}
\subsubsection{Heiltölusegð}
\subsubsection{Fleytitölusegð}
\subsubsection{Stafsegð}
\subsubsection{Strengsegð}
\subsubsection{Listasegð}
\subsubsection{return-segð}
\subsubsection{Röksegðir}
\subsubsection{Kallsegð}
\subsubsection{Tvíundaraðgerðir}
\subsubsection{Einundaraðgerðir}
\subsubsection{if-segð}
\subsubsection{while-segð}
\section{Til hliðsjónar ... hendið þessu eftir notkun}

\subsection{Mállýsing með syntax pakkanum í LaTeX}

Hér er eilítil mállýsing sniðsett með hjálp {\tt syntax} pakkans sem
nota má í \LaTeX. Þið getið notað þessa aðferð, ef það hentar ykkur,
til að skilgeina ykkar forritunarmál.

\begin{grammar}
<x> ::= <x> <y>
	\alt <y> `x'
\end{grammar}

\subsection{Ýmis málrit með hjálp syntax pakkans í LaTeX}

Hér eru nokkur málrit skilgreind með hjálp {\tt syntax} pakkans.
Breytið þessu eftir þörfum, eða notið aðrar
aðferðir ef það hentar ykkur betur.

Finna má upplýsingar um {\tt syntax} pakkann
\htmladdnormallinkfoot{á vefnum}{http://www.tug.org/texlive/Contents/live/texmf-dist/doc/latex/mdwtools/syntax.pdf}.

Takið eftir að ráðlegt er að gera einhverjar breytingar hér,
þó ekki væri til annars en að gera textalýsingar á merkingu
einfaldari með tilliti til forgangs ýmissa málfyrirbæra.

\begin{málrit}{id_list}
	\begin{stack}
	
	\\
		\begin{rep} <id> \\ `,' \end{rep}
	\end{stack}
\end{málrit}%

\subsubsection{Program}

\begin{málrit}{program}
	\begin{repnull}
		\begin{stack}
			<string>
			\begin{stack}
				`=' <module>
			\\
				`=' <id> `in' <module>
			\end{stack}
		\\
			<id> `=' <module>
		\end{stack}
		`;'
	\end{repnull}
\end{málrit}

\subsubsection{Modules}

\begin{málrit}{module}
	\begin{rep}
		\begin{repnull}
			\begin{stack}
				`-'
			\\
				`+'
			\\
				`!'
			\end{stack}
		\end{repnull}
		\begin{stack}
			<string>
		\\
			<id>
		\\
			`(' <module> `)'
		\\
			<textmodule>
		\end{stack}
	\\
		\begin{stack}
			`&'
		\\
			`+'
		\\
			`:'
		\\
			`*'
		\end{stack}
	\end{rep}
\end{málrit}

\begin{málrit}{textmodule}
	`{' `{'
	\begin{stack}
	\\
		\begin{rep}
			<export> `;'
		\end{rep}
	\end{stack}
	`}' `}'
\end{málrit}

\begin{málrit}{export}
	\begin{stack}
		<id>
	\\
		<op>
	\end{stack}
	`='
	\begin{stack}
		`fun' <id> `(' <id_list> `)'
	\\
		`fun' <op> `(' <id_list> `)'
	\\
		`fun' `(' <id_list> `)' <body>
	\end{stack}
\end{málrit}

\subsubsection{Stofnar}
\begin{málrit}{body}
	`{' 
	<decls>
	\begin{repnull} <expr> `;' \end{repnull}
	`}' 
\end{málrit}

\subsubsection{Skilgreiningar}
\begin{málrit}{decls}
	\begin{repnull}
		\begin{stack}
			`var' \begin{rep} <id> \\ `,' \end{rep} `;'
		\\
			`val' \begin{rep} <id> `=' <expr> \\ `,' \end{rep} `;'
		\end{stack}
	\end{repnull}
\end{málrit}

\subsubsection{Segðir}

\begin{málrit}{expr}
	\begin{rep}
		<small_expr>
	\\
		\begin{stack}
			<op>
		\\
			`&&'
		\\
			`||'
		\end{stack}
	\end{rep}
\end{málrit}

\begin{málrit}{small_expr}
	\begin{stack}
		<op> <small_expr>
	\\
		`(' expr `)'
	\\
		<id>
	\\
		<id> `=' <expr>
	\\
		<id> `('
		\begin{repnull}
			<expr>
		\\
			`,'
		\end{repnull}
		`)'
	\\
		`while' `(' expr `)' <body>
	\\
		`return' <expr>
	\\
		`if' `(' expr `)' <body> <if_rest>
	\\
		`['
		\begin{repnull}
			<expr>
		\\
			`,'
		\end{repnull}
		`]'
	\\
		<int>
	\\
		<float>
	\\
		<string>
	\\
		<char>
	\\
		`true'
	\\
		`false'
	\\
		`null'
	\end{stack}
\end{málrit}

\begin{málrit}{if_rest}
	\begin{repnull}
		`elsif' `(' <expr> `)' <body>
	\end{repnull}
	\begin{stack}
	\\
		`else' <body>
	\end{stack}
\end{málrit}

\subsubsection{Annar möguleiki}

\begin{málrit}{expr}
	\begin{stack}
		`return' <expr>
	\\
		<id> `=' <expr>
	\\
		<or_expr>
	\end{stack}
\end{málrit}

\begin{málrit}{or_expr}
	\begin{rep}
		<and_expr>
	\\
		`||'
	\end{rep}
\end{málrit}

\begin{málrit}{and_expr}
	\begin{rep}
		<not_expr>
	\\
		`&&'
	\end{rep}
\end{málrit}

\begin{málrit}{not_expr}
	\begin{repnull}
		`!'
	\end{repnull}
	<binop_expr>
\end{málrit}

\begin{málrit}{binop_expr}
	\begin{rep}
		<small_expr>
	\\
		<op>
	\end{rep}
\end{málrit}

\begin{málrit}{small_expr}
	\begin{stack}
		<op> <small_expr>
	\\
		`(' expr `)'
	\\
		<id>
		\begin{stack}
		\\
		`('
		\begin{repnull}
			<expr>
		\\
			`,'
		\end{repnull}
		`)'
		\end{stack}
	\\
		`while' `(' expr `)' <body>
	\\
		`if' `(' expr `)' <body> <if_rest>
	\\
		`['
		\begin{repnull}
			<expr>
		\\
			`,'
		\end{repnull}
		`]'
	\\
		<int>
	\\
		<float>
	\\
		<string>
	\\
		<char>
	\\
		`true'
	\\
		`false'
	\\
		`null'
	\end{stack}
\end{málrit}

\end{document}
